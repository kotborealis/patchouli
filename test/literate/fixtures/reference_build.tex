\documentclass[12pt,russian,a4paper,,fleqn]{extarticle}
\usepackage{lmodern}
\usepackage{amssymb,amsmath}
\usepackage{ifxetex,ifluatex}
\usepackage{fixltx2e} % provides \textsubscript
\ifnum 0\ifxetex 1\fi\ifluatex 1\fi=0 % if pdftex
  \usepackage[T1]{fontenc}
  \usepackage[utf8]{inputenc}
\else % if luatex or xelatex
  \usepackage{unicode-math}
  \defaultfontfeatures{Ligatures=TeX,Scale=MatchLowercase}
  \newfontfamily{\cyrillicfont}[]{CMU Serif}
  \newfontfamily{\cyrillicfonttt}[Scale=MatchLowercase]{CMU Typewriter
Text}
\fi
% use upquote if available, for straight quotes in verbatim environments
\IfFileExists{upquote.sty}{\usepackage{upquote}}{}
% use microtype if available
\IfFileExists{microtype.sty}{%
\usepackage[]{microtype}
\UseMicrotypeSet[protrusion]{basicmath} % disable protrusion for tt fonts
}{}
\PassOptionsToPackage{hyphens}{url} % url is loaded by hyperref
\usepackage[unicode=true]{hyperref}
\hypersetup{
            pdfborder={0 0 0},
            breaklinks=true}
\urlstyle{same}  % don't use monospace font for urls
\usepackage[margin=2cm]{geometry}
\ifnum 0\ifxetex 1\fi\ifluatex 1\fi=0 % if pdftex
  \usepackage[shorthands=off,main=russian]{babel}
\else
  \usepackage{polyglossia}
  \setmainlanguage[]{russian}
\fi
\usepackage{color}
\usepackage{fancyvrb}
\newcommand{\VerbBar}{|}
\newcommand{\VERB}{\Verb[commandchars=\\\{\}]}
\DefineVerbatimEnvironment{Highlighting}{Verbatim}{commandchars=\\\{\}}
% Add ',fontsize=\small' for more characters per line
\newenvironment{Shaded}{}{}
\newcommand{\AlertTok}[1]{\textcolor[rgb]{1.00,0.00,0.00}{\textbf{#1}}}
\newcommand{\AnnotationTok}[1]{\textcolor[rgb]{0.38,0.63,0.69}{\textbf{\textit{#1}}}}
\newcommand{\AttributeTok}[1]{\textcolor[rgb]{0.49,0.56,0.16}{#1}}
\newcommand{\BaseNTok}[1]{\textcolor[rgb]{0.25,0.63,0.44}{#1}}
\newcommand{\BuiltInTok}[1]{#1}
\newcommand{\CharTok}[1]{\textcolor[rgb]{0.25,0.44,0.63}{#1}}
\newcommand{\CommentTok}[1]{\textcolor[rgb]{0.38,0.63,0.69}{\textit{#1}}}
\newcommand{\CommentVarTok}[1]{\textcolor[rgb]{0.38,0.63,0.69}{\textbf{\textit{#1}}}}
\newcommand{\ConstantTok}[1]{\textcolor[rgb]{0.53,0.00,0.00}{#1}}
\newcommand{\ControlFlowTok}[1]{\textcolor[rgb]{0.00,0.44,0.13}{\textbf{#1}}}
\newcommand{\DataTypeTok}[1]{\textcolor[rgb]{0.56,0.13,0.00}{#1}}
\newcommand{\DecValTok}[1]{\textcolor[rgb]{0.25,0.63,0.44}{#1}}
\newcommand{\DocumentationTok}[1]{\textcolor[rgb]{0.73,0.13,0.13}{\textit{#1}}}
\newcommand{\ErrorTok}[1]{\textcolor[rgb]{1.00,0.00,0.00}{\textbf{#1}}}
\newcommand{\ExtensionTok}[1]{#1}
\newcommand{\FloatTok}[1]{\textcolor[rgb]{0.25,0.63,0.44}{#1}}
\newcommand{\FunctionTok}[1]{\textcolor[rgb]{0.02,0.16,0.49}{#1}}
\newcommand{\ImportTok}[1]{#1}
\newcommand{\InformationTok}[1]{\textcolor[rgb]{0.38,0.63,0.69}{\textbf{\textit{#1}}}}
\newcommand{\KeywordTok}[1]{\textcolor[rgb]{0.00,0.44,0.13}{\textbf{#1}}}
\newcommand{\NormalTok}[1]{#1}
\newcommand{\OperatorTok}[1]{\textcolor[rgb]{0.40,0.40,0.40}{#1}}
\newcommand{\OtherTok}[1]{\textcolor[rgb]{0.00,0.44,0.13}{#1}}
\newcommand{\PreprocessorTok}[1]{\textcolor[rgb]{0.74,0.48,0.00}{#1}}
\newcommand{\RegionMarkerTok}[1]{#1}
\newcommand{\SpecialCharTok}[1]{\textcolor[rgb]{0.25,0.44,0.63}{#1}}
\newcommand{\SpecialStringTok}[1]{\textcolor[rgb]{0.73,0.40,0.53}{#1}}
\newcommand{\StringTok}[1]{\textcolor[rgb]{0.25,0.44,0.63}{#1}}
\newcommand{\VariableTok}[1]{\textcolor[rgb]{0.10,0.09,0.49}{#1}}
\newcommand{\VerbatimStringTok}[1]{\textcolor[rgb]{0.25,0.44,0.63}{#1}}
\newcommand{\WarningTok}[1]{\textcolor[rgb]{0.38,0.63,0.69}{\textbf{\textit{#1}}}}
\usepackage{graphicx,grffile}
\makeatletter
\def\maxwidth{\ifdim\Gin@nat@width>\linewidth\linewidth\else\Gin@nat@width\fi}
\def\maxheight{\ifdim\Gin@nat@height>\textheight\textheight\else\Gin@nat@height\fi}
\makeatother
% Scale images if necessary, so that they will not overflow the page
% margins by default, and it is still possible to overwrite the defaults
% using explicit options in \includegraphics[width, height, ...]{}
\setkeys{Gin}{width=\maxwidth,height=\maxheight,keepaspectratio}
\IfFileExists{parskip.sty}{%
\usepackage{parskip}
}{% else
\setlength{\parindent}{0pt}
\setlength{\parskip}{6pt plus 2pt minus 1pt}
}
\setlength{\emergencystretch}{3em}  % prevent overfull lines
\providecommand{\tightlist}{%
  \setlength{\itemsep}{0pt}\setlength{\parskip}{0pt}}
\setcounter{secnumdepth}{0}
% Redefines (sub)paragraphs to behave more like sections
\ifx\paragraph\undefined\else
\let\oldparagraph\paragraph
\renewcommand{\paragraph}[1]{\oldparagraph{#1}\mbox{}}
\fi
\ifx\subparagraph\undefined\else
\let\oldsubparagraph\subparagraph
\renewcommand{\subparagraph}[1]{\oldsubparagraph{#1}\mbox{}}
\fi

% set default figure placement to htbp
\makeatletter
\def\fps@figure{htbp}
\makeatother


\date{}

\usepackage{float}
\let\origfigure=\figure
\let\endorigfigure=\endfigure
\renewenvironment{figure}[1][]{%
\origfigure[H]
}{%
\endorigfigure
}

% My settings
\usepackage{caption}
\usepackage{mathtools}
\usepackage{pdfpages}
\usepackage{amsthm}
\usepackage{unicode-math}
\usepackage{unicode-math}
\usepackage{listings}

% Theorems, proofs, etc
\newtheorem{theorem}{Теорема}
\newtheorem{corollary}{Следствие}[theorem]
\newtheorem{lemma}[theorem]{Лемма}
\theoremstyle{definition}\newtheorem{definition}{Определение}
\theoremstyle{remark}\newtheorem*{remark}{Примечание}
% ----

% equation numbering
\numberwithin{equation}{section}

\captionsetup[figure]{name=Рисунок}
\captionsetup[table]{name=Таблица}

% listings
\lstset{
    breaklines=true,
    captionpos=t,
    numbers=left,
    inputencoding=utf8x,
    extendedchars=\true,
    keepspaces=\true
}
\renewcommand*{\lstlistingname}{Листинг}
\newcommand{\includecode}[2][c]{\lstinputlisting[caption=#2, language=#1]{#2}}
\newcommand{\includecodelines}[4][c]{\lstinputlisting[caption=#2 стр. #3–#4, firstnumber=#3, firstline=#3, lastline=#4, language=#1]{#2}}

% commands
\newcommand{\locmin}{\operatorname{locmin}}
\newcommand{\locmax}{\operatorname{locmax}}
\newcommand{\absmin}{\operatorname{absmin}}
\newcommand{\absmax}{\operatorname{absmax}}
\newcommand{\locextr}{\operatorname{locmax}}
\newcommand{\extr}{\operatorname{extr}}
\newcommand{\ohat}[1]{\hat{\overline{#1}}}
\newcommand{\dpartial}[2]{\dfrac{\partial #1}{\partial #2}}
\newcommand{\grad}{\operatorname{grad}}
\newcommand{\epi}{\operatorname{epi}}

\begin{document}

\section{Literate examples}

Define function:

\begin{Shaded}
\begin{Highlighting}[]
\KeywordTok{this}\OperatorTok{.}\AttributeTok{fn} \OperatorTok{=}\NormalTok{ (x) }\KeywordTok{=\textgreater{}} \BuiltInTok{Math}\OperatorTok{.}\FunctionTok{sin}\NormalTok{(x)}\OperatorTok{;}
\end{Highlighting}
\end{Shaded}

Return value:

\begin{Shaded}
\begin{Highlighting}[]
\ControlFlowTok{return}\NormalTok{ fn(}\DecValTok{0}\NormalTok{)}\OperatorTok{;}
\end{Highlighting}
\end{Shaded}

0

Return string:

\begin{Shaded}
\begin{Highlighting}[]
\ControlFlowTok{return} \StringTok{"Hello, world!"}\OperatorTok{;}
\end{Highlighting}
\end{Shaded}

Hello, world!

Print something:

\begin{Shaded}
\begin{Highlighting}[]
\BuiltInTok{console}\OperatorTok{.}\FunctionTok{log}\NormalTok{(}\DecValTok{1}\OperatorTok{,}\DecValTok{2}\OperatorTok{,}\DecValTok{3}\NormalTok{)}\OperatorTok{;}
\BuiltInTok{console}\OperatorTok{.}\FunctionTok{log}\NormalTok{(}\StringTok{"Hello"}\OperatorTok{,} \StringTok{"world"}\NormalTok{)}\OperatorTok{;}
\end{Highlighting}
\end{Shaded}

1 2 3 Hello world

Print something and return:

\begin{Shaded}
\begin{Highlighting}[]
\BuiltInTok{console}\OperatorTok{.}\FunctionTok{log}\NormalTok{(\{}\DataTypeTok{a}\OperatorTok{:} \DecValTok{1}\OperatorTok{,} \DataTypeTok{b}\OperatorTok{:} \StringTok{"Кириллица?"}\NormalTok{\})}\OperatorTok{;}
\ControlFlowTok{return} \BuiltInTok{Math}\OperatorTok{.}\ConstantTok{PI}\OperatorTok{;}
\end{Highlighting}
\end{Shaded}

\{ a: 1, b: `Кириллица?' \} 3.141592653589793

Do not eval code block:

\begin{Shaded}
\begin{Highlighting}[]
\BuiltInTok{console}\OperatorTok{.}\FunctionTok{log}\NormalTok{(\{}\DataTypeTok{a}\OperatorTok{:} \DecValTok{1}\OperatorTok{,} \DataTypeTok{b}\OperatorTok{:} \StringTok{"Кириллица?"}\NormalTok{\})}\OperatorTok{;}
\ControlFlowTok{return} \BuiltInTok{Math}\OperatorTok{.}\ConstantTok{PI}\OperatorTok{;}
\end{Highlighting}
\end{Shaded}

Do not print code block, but print results:

\{ a: 1, b: `Кириллица?' \} 3.141592653589793

Plot something:

\begin{Shaded}
\begin{Highlighting}[]
\KeywordTok{const}\NormalTok{ plotter }\OperatorTok{=} \KeywordTok{new}\NormalTok{ Plotter}\OperatorTok{;}
\ControlFlowTok{return} \ControlFlowTok{await}\NormalTok{ plotter}\OperatorTok{.}\FunctionTok{plot}\NormalTok{(\{}\DataTypeTok{data}\OperatorTok{:}\NormalTok{ [[}\DecValTok{0}\OperatorTok{,}\DecValTok{0}\NormalTok{]}\OperatorTok{,}\NormalTok{ [}\DecValTok{10}\OperatorTok{,}\DecValTok{10}\NormalTok{]}\OperatorTok{,}\NormalTok{ [}\DecValTok{20}\OperatorTok{,}\DecValTok{20}\NormalTok{]]\})}\OperatorTok{;}
\end{Highlighting}
\end{Shaded}

\includegraphics{/tmp/tmp-4883-J4RUdCJnHlD3-.pdf}

Plot with reference to it (see \ref{fig:plot-example}):

\begin{Shaded}
\begin{Highlighting}[]
\KeywordTok{const}\NormalTok{ plotter }\OperatorTok{=} \KeywordTok{new}\NormalTok{ Plotter}\OperatorTok{;}
\KeywordTok{const}\NormalTok{ plot }\OperatorTok{=} \ControlFlowTok{await}\NormalTok{ plotter}\OperatorTok{.}\FunctionTok{splot}\NormalTok{(\{}\DataTypeTok{data}\OperatorTok{:}\NormalTok{ [[}\DecValTok{0}\OperatorTok{,}\DecValTok{0}\OperatorTok{,}\DecValTok{0}\NormalTok{]}\OperatorTok{,}\NormalTok{ [}\DecValTok{10}\OperatorTok{,}\DecValTok{10}\OperatorTok{,}\DecValTok{20}\NormalTok{]}\OperatorTok{,}\NormalTok{ [}\DecValTok{20}\OperatorTok{,}\DecValTok{20}\OperatorTok{,}\DecValTok{100}\NormalTok{]]\})}\OperatorTok{;}
\NormalTok{plot}\OperatorTok{.}\AttributeTok{caption} \OperatorTok{=} \StringTok{"Example of plot with caption \& label"}\OperatorTok{;}
\NormalTok{plot}\OperatorTok{.}\AttributeTok{label} \OperatorTok{=} \StringTok{"plot{-}example"}\OperatorTok{;}
\ControlFlowTok{return}\NormalTok{ plot}\OperatorTok{;}
\end{Highlighting}
\end{Shaded}

\begin{figure}
\hypertarget{fig:plot-example}{%
\centering
\includegraphics{/tmp/tmp-4883-OrfO27ei72VU-.pdf}
\caption{Example of plot with caption \& label}\label{fig:plot-example}
}
\end{figure}

Can we have a table? (see \ref{table:example-table})

\begin{Shaded}
\begin{Highlighting}[]
\KeywordTok{const}\NormalTok{ data }\OperatorTok{=}\NormalTok{ [}
\NormalTok{   [}\StringTok{"Oh hi mark"}\OperatorTok{,} \StringTok{"$}\SpecialCharTok{\textbackslash{}\textbackslash{}}\StringTok{omega$"}\OperatorTok{,} \StringTok{"!"}\NormalTok{]}\OperatorTok{,}
\NormalTok{   [}\DecValTok{1}\OperatorTok{,} \DecValTok{2}\OperatorTok{,} \DecValTok{3}\NormalTok{]}\OperatorTok{,}
\NormalTok{   [}\DecValTok{4}\OperatorTok{,} \DecValTok{5}\OperatorTok{,} \DecValTok{6}\NormalTok{]}\OperatorTok{,}
\NormalTok{   [}\DecValTok{7}\OperatorTok{,} \DecValTok{8}\OperatorTok{,} \DecValTok{9}\NormalTok{]}
\NormalTok{]}\OperatorTok{;}

\KeywordTok{const}\NormalTok{ table }\OperatorTok{=} \KeywordTok{new}\NormalTok{ BlockTable(data)}\OperatorTok{;}
\NormalTok{table}\OperatorTok{.}\AttributeTok{caption} \OperatorTok{=} \StringTok{"Example of table with caption \& label"}
\NormalTok{table}\OperatorTok{.}\AttributeTok{label} \OperatorTok{=} \StringTok{"example{-}table"}
\ControlFlowTok{return}\NormalTok{ table}
\end{Highlighting}
\end{Shaded}

\begin{table}[ht]
\centering
\begin{tabular}[t]{|c|c|c|}
\hline
Oh hi mark & $\omega$ & !\\
\hline
1 & 2 & 3\\
\hline
4 & 5 & 6\\
\hline
7 & 8 & 9\\
\hline
\end{tabular}
\caption{
\label{table:example-table}
Example of table with caption \& label}
\end{table}

\end{document}
